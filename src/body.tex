\section{Introduction}

Producing reward signal is a fundamental brain system for animals to live and reproduce. Positive rewards such as pleasure encourages us to eat and find partners, while negative rewards such as fatigue are useful for protecting ourselves. An important function of the reward signal is to give a direction to our learning and behavior. Our brain produces reward signal to a certain stimulus such as good food, and the signal is used to reinforce the behavior that leads to the reward. Due to its importance, the learning based on reward, called reinforcement learning, has been extensively studied in both neuroscience and computer science. Neuroscientists have revealed that our brain has some hot or cold spots that respond to good or bad events~\cite{berridgeAffectiveNeurosciencePleasure2008}, and how those signals are used for learning~\cite{schultzNeuronalRewardDecision2015}, while computational reinforcement learning has provided theoretical models and several applications including game-playing agents and robotics.

However, compared to reinforcement learning, less attention has been paid to why we have such rewards, and how we had acquired it. This may be partly because the role played by the reward system in evolution seems so obvious. A common explanation is based on natural selection, arguing that rewards have evolved to help animals survive in the environment and successfully reproduce offspring (e.g., by~\cite{schultzNeuronalRewardDecision2015}). While this hypothesis is reasonable, there remain some questions in the detailed process of reward evolution. An important
Specifically, in this paper, we focus on the question of how environmental traits and ecological dynamics affects evolved rewards.

\section{Related Work}