%% main file for emevo paper
%% The first command in your LaTeX source must be the \documentclass command.
\documentclass[sigconf]{acmart}

%% Rights management information.
% \acmDOI{10.1145/nnnnnnn.nnnnnnn}
% \acmISBN{978-x-xxxx-xxxx-x/YY/MM}
% \acmConference[GECCO '23]{The Genetic and Evolutionary Computation Conference 2023}{July 15--19, 2023}{Lisbon, Portugal}
% \acmYear{2023}
% \copyrightyear{2023}

%% end of the preamble, start of the body of the document source.
\begin{document}

%%
%% The "title" command has an optional parameter,
%% allowing the author to define a "short title" to be used in page headers.
\title{Embodied Evolution}

\author{Yuji Kanagawa}
\orcid{0000-0002-0925-3894}
\affiliation{%
  \institution{Okinawa Institute of Science and Technology Graduate University}
  \state{Okinawa}
  \country{Japan}
}
\email{yuji.kanagawa@oist.jp}

\author{Kenji Doya}
\affiliation{%
  \institution{Okinawa Institute of Science and Technology Graduate University}
  \state{Okinawa}
  \country{Japan}
}
\email{doya@oist.jp}

\begin{abstract}
  The reward system is one of the fundamental drivers for animals to live.
\end{abstract}

%%
%% The code below is generated by the tool at http://dl.acm.org/ccs.cfm.
%% Please copy and paste the code instead of the example below.
%%
\begin{CCSXML}
<ccs2012>
   <concept>
       <concept_id>10003752.10010070.10010071.10010261.10010275</concept_id>
       <concept_desc>Theory of computation~Multi-agent reinforcement learning</concept_desc>
       <concept_significance>300</concept_significance>
       </concept>
   <concept>
       <concept_id>10010147.10010257.10010293.10011809.10011814</concept_id>
       <concept_desc>Computing methodologies~Evolutionary robotics</concept_desc>
       <concept_significance>500</concept_significance>
       </concept>
 </ccs2012>
\end{CCSXML}

\ccsdesc[500]{Computing methodologies~Evolutionary robotics}
\ccsdesc[300]{Theory of computation~Multi-agent reinforcement learning}

\keywords{embodied evolution, evolutionary robotics, reinforcement learning}

%% A "teaser" image appears between the author and affiliation
%% information and the body of the document, and typically spans the
%% page.
% \begin{teaserfigure}
%   \includegraphics[width=\textwidth]{sampleteaser}
%   \caption{Seattle Mariners at Spring Training, 2010.}
%   \Description{Enjoying the baseball game from the third-base seats. Ichiro Suzuki preparing to bat.}
%   \label{fig:teaser}
% \end{teaserfigure}

\maketitle

\section{Introduction}

Rewards are fundamental drivers for animals to live and reproduce. Positive rewards guide individuals to do the things necessary to live and produce offspring, such as eating food and mating. Negative rewards such as pain help us protect against injuries. Rewards trigger instinctive behaviors, create good moods, and reinforce previously rewarded behaviors. Thus, producing rewards in response to relevant sensory inputs is a vital function of our brains, and some brain regions do it. There are brain regions that judge whether the sensory inputs are rewarding or not \cite{schultzNeuronalRewardDecision2015}. The entire network involved in the reward pathway is called the reward system.
It discriminates rewards, triggers rewarding actions, and initiates learning of behavioral strategies that may result in greater rewards. In addition to subjective feelings, this characteristic of inducing learning makes the difference between reward-based behavior and innate behavior.

Along with academic interest, rewards have also been studied extensively for medical applications, such as treatment for  addiction \cite{koobNeuroscienceAddiction1998,solinasDopamineAddictionWhat2019}, depression \cite{dunlopRoleDopaminePathophysiology2007}, and other psychological disorders.
Drugs that act on the reward system, such as dopamine antagonists, are widely prescribed commercially for the treatment of psychiatric disorders and ADHD, while they are not perfect, making the study on rewards important. However, less attention has been paid to why we have such rewards. Perhaps this is because the role played by the reward system in evolution seems so obvious. A common explanation is based on natural selection, arguing that rewards have evolved to help animals survive in the environment and successfully reproduce offspring (e.g., by \cite{schultzNeuronalRewardDecision2015}).

\section{Template Overview}

\bibliographystyle{ACM-Reference-Format}
\bibliography{references}

\appendix

\section{Research Methods}

\subsection{Part One}

\section{Online Resources}


\end{document}
\endinput
