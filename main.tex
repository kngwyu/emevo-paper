\documentclass[10pt]{article} % For LaTeX2e
\usepackage{rlc}

% If accepted, instead use the following line for the camera-ready submission:
% \usepackage[accepted]{rlc}
% To de-anonymize and remove mentions to RLC (for example, for posting to preprint servers), instead use the following:
% \usepackage[preprint]{rlc}

%%%%%%%%%%%%%%%%%%%%%%%%%%%%%%%%%%%%%%%%%%%%%%%%%%%%%%%%%%%%%%%%
%% Recommended (but not required) packages
%%%%%%%%%%%%%%%%%%%%%%%%%%%%%%%%%%%%%%%%%%%%%%%%%%%%%%%%%%%%%%%%
\usepackage{amssymb}            % Defines common symbols like \mathbb R
\usepackage{mathtools}          % Extends amsmath, providing common math tools
\usepackage{mathrsfs}           % Enables \mathscr, which can work in cases that \mathcal does not
\usepackage{graphicx}           % For including images
\usepackage[bf]{caption} % caption: FIG in bold
\usepackage{subcaption}         % Allows for the use of subfigures and subcaptions
\usepackage[space]{grffile}     % For spaces in image names
\usepackage{url}                % For displaying urls
\usepackage{cleveref}

%%%%%% Cut padding
\setlength{\belowcaptionskip}{-10pt}
\setlength{\abovedisplayskip}{1pt}
\setlength{\belowdisplayskip}{3pt}
%%%%% End: Cut padding

%%%%%% Todo utilities
\usepackage{todonotes}
\setuptodonotes{
  inline,
  linecolor=blue,
  backgroundcolor=blue,
  textcolor=white,
  bordercolor=blue,
}
%%%%% End: Todo utilities

%%%%% Math
\newcommand{\defequal}{\mathrel{\overset{\makebox[0pt]{\mbox{\tiny def}}}{=}}}
\newcommand{\ubcomment}[2]{\underbrace{#1}_{\mbox{\small #2}}}
\newcommand{\sample}[1]{\ubcomment{#1}{sample}}
\newcommand{\continuation}[1]{\ubcomment{#1}{sample(continuation)}}
\newcommand{\E}{\mathbb{E}} % Expectation
\newcommand{\1}{\mathbb{I} } % indicator with no arguments
\newcommand{\indic}[1]{\1_{#1}} % indicator function
\DeclareMathOperator*{\argmax}{\arg\max}
\DeclareMathOperator*{\argmin}{\arg\min}
\newcommand{\KL}{D_{\mathrm{KL}}}
%%%%% End: Math

%%%%% Roman number
\newcounter{num}
\newcommand{\rnum}[1]{\setcounter{num}{#1}(\roman{num})}
%%%%% End: Roman number

%%%%% MDP
\newcommand{\M}{\mathcal{M}} % MDP
\newcommand{\A}{\mathcal{A}} % Actions
\newcommand{\X}{\mathcal{S}} % States
\newcommand{\ah}{\hat{A}} % Estimation of Advantage
\newcommand{\vh}{\hat{V}} % Estimation of Value
%%%%% End: MDP

%%%%% Table
\usepackage{siunitx}
\usepackage{booktabs}
\usepackage{longtable} % tables that can span several pages
%%%%% End: Table

%%%%% Graphics
\graphicspath{{./resources}} % Specifies the directory where pictures are stored
%%%%% End: Graphics

%%%%% Pseudo codes
\usepackage{algpseudocodex}
\usepackage{algorithm}
\makeatletter
\algnewcommand\algorithmiconce{\textbf{Once}}%
\algdef{SE}[ONCE]{Once}{EndOnce}[1]{\algpx@startIndent\algpx@startCodeCommand\algorithmiconce\ #1\ \algorithmicdo%
}{\algpx@endIndent\algpx@startEndBlockCommand\algorithmicend\ \algorithmiconce}%
\pretocmd{\Once}{\algpx@endCodeCommand}{}{}
\algtext*{EndOnce}%
\apptocmd{\EndOnce}{\algpx@endIndent}{}{}%
\pretocmd{\EndOnce}{\algpx@endCodeCommand[1]}{}{}%

\algnewcommand\algorithmicWith{\textbf{With}}%
\algdef{SE}[WITH]{With}{EndWith}[1]{\algpx@startIndent\algpx@startCodeCommand\algorithmicWith\ #1\ \algorithmicdo%
}{\algpx@endIndent\algpx@startEndBlockCommand\algorithmicend\ \algorithmicWith}%
\pretocmd{\With}{\algpx@endCodeCommand}{}{}
\algtext*{EndWith}%
\apptocmd{\EndWith}{\algpx@endIndent}{}{}%
\pretocmd{\EndWith}{\algpx@endCodeCommand[1]}{}{}%
\makeatother
%%%%% End: Pseudo Codes

%%%%%%%%%%%%%%%%%%%%%%%%%%%%%%%%%%%%%%%%%%%%%%%%%%%%%%%%%%%%%%%%
%% Title Page Specification
%%%%%%%%%%%%%%%%%%%%%%%%%%%%%%%%%%%%%%%%%%%%%%%%%%%%%%%%%%%%%%%%
\title{Emergence of Food Reward with Satiety in Decentralized Evolution of Reward Functions}

% Authors must not appear in the submitted version. They should be hidden
% as long as the tmlr package is used without the [accepted] or [preprint] options.
% Non-anonymous submissions will be rejected without review.

\author{Yuji Kanagawa  \\
    yuji.kanagawa@oist.jp \\
    Okinawa Institute of Science and Technology Graduate University \\
    \And
    Kenji Doya \\
    doya@oist.jp\\
    Okinawa Institute of Science and Technology Graduate University
}

% The \author macro works with any number of authors. Use \AND
% to separate the names and addresses of multiple authors.

%%%%%%%%%%%%%%%%%%%%%%%%%%%%%%%%%%%%%%%%%%%%%%%%%%%%%%%%%%%%%%%%
%% Begin document, create title and abstract
%%%%%%%%%%%%%%%%%%%%%%%%%%%%%%%%%%%%%%%%%%%%%%%%%%%%%%%%%%%%%%%%
\begin{document}

\maketitle

\begin{abstract}
  The reward system is one of the fundamental drivers for animals to live, motivating behaviors required for survival and reproduction and serving as objective for reinforcement learning. Despite its importance, the problem of how the reward system has evolved is underexplored. In this paper, we study in what environmental conditions biologically plausible reward functions can evolve. For this purpose, we employ a decentralized evolutionary simulation framework where each agent inherits its reward function from its parent and learns to get more rewards via reinforcement learning throughout its lifetime. Our simulation results show that the evolution of food reward is often unstable under many conditions because the overcosumpition of foods
%by a few agents
  leads to population collapse. We also show that food reward is more stable with the reward functions depending on the agent's nutritional status %regulates food reward
  , suggesting that satiety is required to evolve sustainable food reward.
\end{abstract}

\section{Introduction}

Producing reward signal is a fundamental brain system for animals to live and reproduce. Positive rewards such as pleasure encourages us to eat and find partners, while negative rewards such as fatigue are useful for protecting ourselves. An important function of the reward signal is to give a direction to our learning and behavior. Our brain produces reward signal to a certain stimulus such as good food, and the signal is used to reinforce the behavior that leads to the reward. Due to its importance, the learning based on reward, called reinforcement learning (RL), has been extensively studied in both neuroscience and computer science. Neuroscientists have revealed that our brain has some hot or cold spots that respond to good or bad events~\cite{berridgeAffectiveNeurosciencePleasure2008}, and how those signals are used for learning~\cite{schultzNeuronalRewardDecision2015}, while computational reinforcement learning has provided theoretical models and several applications including game-playing agents and robotics.

However, compared to RL, less attention has been paid to the reward itself. Why we have such rewards, what is the common traits, and how did we acquired it in the course of evolution? This may be partly because the role played by the reward system in evolution seems obvious. A common explanation is based on natural selection, arguing that rewards have evolved to help animals survive in the environment and successfully reproduce offspring (e.g., by~\cite{schultzNeuronalRewardDecision2015}). While this hypothesis is reasonable, there remain some questions in the detailed process of reward evolution. An important questions include in what environmental conditions rewards are important. For example, in an environment where animals can get enough nutrition without doing anything, maybe they don't need any reward for food. Thus, the need of foraging and competition with other individuals can make food rewards important, but their effect is not obvious. \todo{One more question}

To answer these questions, we propose an embodied evolution (EE)~\cite{watsonEmbodiedEvolutionDistributing2002} model of reward functions. While real experiments of evolution and phylogenetic analysis are highly important, it is still not plausible to conduct extensive evolutionary experiments using animals with brains complex enough to have reward systems. Thus, computer simulation might be a good choice in that it can enable us simulate the relatively long course of evolution, while we need to sacrifice some kind of reality. Since our aim is to confirm the environmental effects to reward functions in a biologically plausible way, EE is a good coice.

We implemented our model by JAX.

\section{Related Work}
My work is mainly inspired by the series of studies by \citet{elfwingBiologicallyInspiredEmbodied2005,elfwingDarwinianEmbodiedEvolution2011a,elfwingEmergencePolymorphicMating2014}. They tried to evolve shaping rewards and meta parameters of reinforcement learning agents using mobile robots called \textit{Cyber Rodents}. In particular, \citet{elfwingDarwinianEmbodiedEvolution2011a} evolved reward-shaping parameters of RL agents, showing that rewards that make learning easier had evolved. Furthermore, \citet{elfwingEmergencePolymorphicMating2014} evolved agents with a hierarchical RL mechanism, where the higher module chooses either foraging or mating, and the lower module acts based on, the higher choice (i.e., foraging or mating) and the current state. They evolved only the higher decision-making parameters, while the lower modules were learned by RL\@. In their experiments, they observed two different groups: one that favors mating behavior and the other that favors foraging behavior. In our study, we try to evolve primary rewards instead of shaping rewards or the hierarchical decision-making mechanism they use. \citet{uchibeFindingIntrinsicRewards2008} also experimented with a similar mobile robot where primitive rewards such as food were predefined, but intrinsic rewards such as curiosity were evolved. In contrast, our study attempts to evolve primitive rewards. A drawback of EE is that reproduction is often done by replacing the genome of the same robot \citep{bredecheEmbodiedEvolutionCollective2018}. This makes it challenging to reproduce population increases and decreases by simulation. While the Darwinian EE framework proposed by \citet{elfwingDarwinianEmbodiedEvolution2011a} partially addressed this by having multiple agents as a virtual population inside a robot, we allow for population increase and decrease in simulation.

\subsubsection*{Broader Impact Statement}
\label{sec:broaderImpact}
In this optional section, RLC encourages authors to discuss possible repercussions of their work, notably any potential negative impact that a user of this research should be aware of.

\subsubsection*{Acknowledgments}
\label{sec:ack}
Use unnumbered third level headings for the acknowledgments. All acknowledgments, including those to funding agencies, go at the end of the paper. Only add this information once your submission is accepted and deanonymized.

%%%%%%%%%%%%%%%%%%%%%%%%%%%%%%%%%%%%%%%%%%%%%%%%%%%%%%%%%%%%%%%%
%% Bibliography
%%%%%%%%%%%%%%%%%%%%%%%%%%%%%%%%%%%%%%%%%%%%%%%%%%%%%%%%%%%%%%%%
\bibliography{references}
\bibliographystyle{rlc}

%%%%%%%%%%%%%%%%%%%%%%%%%%%%%%%%%%%%%%%%%%%%%%%%%%%%%%%%%%%%%%%%
%% Appendices
%%%%%%%%%%%%%%%%%%%%%%%%%%%%%%%%%%%%%%%%%%%%%%%%%%%%%%%%%%%%%%%%
\appendix
\section{Physics Simulation}\label{ap:phys}
Our simulator implements projected Gauss-Seidel method with position correction \citep{catto2005iterative} that is fairly common in 2D game physics engines such as Box2D\footnote{\url{https://box2d.org}} and Chipmunk\footnote{\url{https://chipmunk-physics.net}}.

\section{Parameters used in experiments}\label{ap:param}

\begin{table}[t]
  \centering
  \caption{Parameters for $h$ and $e$}\label{tab:bd-param}
  \begin{tabular}{ccl}
  \toprule
    Parameter & Value & Description \\
    \midrule
    $\kappa_{h}$ & 0.1 & Scaling parameter of the energy-related term of the hazard function \\
    $\alpha_{he}$ & 0.3 & Inverse initial value of the energy-related term of the hazard function \\
    $d_{h}$ & 0.0 & Delay of energy-related term of the hazard function \\
    $\alpha_{ht}$ & \num{1e-6} & Inverse initial value of the age-related term of the hazard function \\
    $\beta$ & \num{1e-6} & Inverse initial value of the age-related term of the hazard function \\
    $\kappa_{b}$ & \num{2e-5} & Scaling parameter of the birth function\\
    $\alpha_{b}$ & 0.1 & Inverse initial value of the birth function \\
    $d_{b}$ & 5.0 & Delay of the birth function \\
    \bottomrule
  \end{tabular}
\end{table}

\begin{table}[t]
  \centering
  \caption{RL parameters}\label{tab:rl-param}
  \begin{tabular}{ccl}
  \toprule
    Parameter & Value & Description \\
    \midrule
    $\gamma$ & 0.995 & Discount factor \\
    $N$ & 1024 & Rollout steps \\
    $N_{B}$ & 8 & PPO Num. minibatches \\
    $N_{E}$ & 4 & PPO Optimization epochs \\
    $\epsilon$ & 0.2 & PPO Clip epsilon \\
    $c_{H}$ & 0.005 & PPO entropy regulization coefficient \\
  \end{tabular}
\end{table}


\end{document}
\endinput
